\subsection{Network class}
\label{Network}
A Network is constructed by adding individual neurons synapses to the network. Neurons are given indices (from 0) which should be unique for each neuron. When adding synapses the source or target neurons need not necessarily exist yet, but should be defined before the network is finalised.

\noindent C++:
\begin{lstlisting}[aboveskip=2pt]
Network::Network()
\end{lstlisting}

\noindent C:
\begin{lstlisting}[aboveskip=2pt]
nemo_network_t
nemo_new_network()
 
nemo_delete_network(nemo_network_t net)
\end{lstlisting}

\noindent Matlab:
\begin{lstlisting}[aboveskip=2pt]
net = nemoNetwork()
\end{lstlisting}
\paragraph{Functions}
\begin{itemize}
\item \hyperref[fn: addNeuron]{addNeuron}
\item \hyperref[fn: addSynapse]{addSynapse}
\item \hyperref[fn: neuronCount]{neuronCount}
\item \hyperref[fn: clearNetwork]{clearNetwork}
\end{itemize}
\clearpage
\subsubsection*{Network::addNeuron}
\label{fn: addNeuron}
add a single neuron to network


\noindent C++:
\begin{lstlisting}[aboveskip=2pt]
void
Network::addNeuron(unsigned idx, float a, float b, float c, 
        float d, float u, float v, float sigma)
\end{lstlisting}

\noindent C:
\begin{lstlisting}[aboveskip=2pt]
nemo_status_t
nemo_add_neuron(nemo_network_t net, unsigned idx, float a, 
        float b, float c, float d, float u, float v, float sigma)
\end{lstlisting}

\noindent Matlab:
\begin{lstlisting}[aboveskip=2pt]
net.addNeuron(idx, a, b, c, d, u, v, sigma)
\end{lstlisting}
\paragraph{Inputs}
\begin{description}
\item[idx] Neuron index (0-based)
\item[a] Time scale of the recovery variable
\item[b] Sensitivity to sub-threshold fluctuations in the membrane potential v
\item[c] After-spike value of the membrane potential v
\item[d] After-spike reset of the recovery variable u
\item[u] Initial value for the membrane recovery variable
\item[v] Initial value for the membrane potential
\item[sigma] Parameter for a random gaussian per-neuron process which generates random input current drawn from an N(0, sigma) distribution. If set to zero no random input current will be generated
\end{description}
The neuron uses the Izhikevich neuron model. See E. M. Izhikevich "Simple model of spiking neurons", IEEE Trans. Neural Networks, vol 14, pp 1569-1572, 2003 for a full description of the model and the parameters.

\clearpage
\subsubsection*{Network::addSynapse}
\label{fn: addSynapse}
add a single synapse to the network


\noindent C++:
\begin{lstlisting}[aboveskip=2pt]
uint64_t
Network::addSynapse(unsigned source, unsigned target, 
        unsigned delay, float weight, unsigned char plastic)
\end{lstlisting}

\noindent C:
\begin{lstlisting}[aboveskip=2pt]
nemo_status_t
nemo_add_synapse(nemo_network_t net, unsigned source, 
        unsigned target, unsigned delay, float weight, 
        unsigned char plastic, uint64_t* id)
\end{lstlisting}

\noindent Matlab:
\begin{lstlisting}[aboveskip=2pt]
id = net.addSynapse(source, target, delay, weight, plastic)
\end{lstlisting}
\paragraph{Inputs}
\begin{description}
\item[source] Index of source neuron
\item[target] Index of target neuron
\item[delay] Synapse conductance delay in milliseconds
\item[weight] Synapse weights
\item[plastic] Boolean specifying whether or not this synapse is plastic
\end{description}
\paragraph{Outputs}
\begin{description}
\item[id] Unique synapse ID
\end{description}

\clearpage
\subsubsection*{Network::neuronCount}
\label{fn: neuronCount}



\noindent C++:
\begin{lstlisting}[aboveskip=2pt]
unsigned
Network::neuronCount()
\end{lstlisting}

\noindent C:
\begin{lstlisting}[aboveskip=2pt]
nemo_status_t
nemo_neuron_count(nemo_network_t net, unsigned* ncount)
\end{lstlisting}

\noindent Matlab:
\begin{lstlisting}[aboveskip=2pt]
ncount = net.neuronCount()
\end{lstlisting}
\paragraph{Outputs}
\begin{description}
\item[ncount] number of neurons in the network
\end{description}

\clearpage
\subsubsection*{Network::clearNetwork}
\label{fn: clearNetwork}
clear all neurons/synapses from network


\noindent C++:
\begin{lstlisting}[aboveskip=2pt]
void
Network::clearNetwork()
\end{lstlisting}

\noindent C:
\begin{lstlisting}[aboveskip=2pt]
nemo_status_t
nemo_clear_network(nemo_network_t net)
\end{lstlisting}

\noindent Matlab:
\begin{lstlisting}[aboveskip=2pt]
net.clearNetwork()
\end{lstlisting}

\clearpage
\subsection{Configuration class}
\label{Configuration}

\noindent C++:
\begin{lstlisting}[aboveskip=2pt]
Configuration::Configuration()
\end{lstlisting}

\noindent C:
\begin{lstlisting}[aboveskip=2pt]
nemo_configuration_t
nemo_new_configuration()
 
nemo_delete_configuration(nemo_configuration_t conf)
\end{lstlisting}

\noindent Matlab:
\begin{lstlisting}[aboveskip=2pt]
conf = nemoConfiguration()
\end{lstlisting}
\paragraph{Functions}
\begin{itemize}
\item \hyperref[fn: setCpuBackend]{setCpuBackend}
\item \hyperref[fn: setCudaBackend]{setCudaBackend}
\item \hyperref[fn: setStdpFunction]{setStdpFunction}
\item \hyperref[fn: backendDescription]{backendDescription}
\item \hyperref[fn: resetConfiguration]{resetConfiguration}
\end{itemize}
\clearpage
\subsubsection*{Configuration::setCpuBackend}
\label{fn: setCpuBackend}
specify that the CPU backend should be used


\noindent C++:
\begin{lstlisting}[aboveskip=2pt]
void
Configuration::setCpuBackend(int tcount)
\end{lstlisting}

\noindent C:
\begin{lstlisting}[aboveskip=2pt]
nemo_status_t
nemo_set_cpu_backend(nemo_configuration_t conf, int tcount)
\end{lstlisting}

\noindent Matlab:
\begin{lstlisting}[aboveskip=2pt]
conf.setCpuBackend(tcount)
\end{lstlisting}
\paragraph{Inputs}
\begin{description}
\item[tcount] number of threads
\end{description}
Specify that the CPU backend should be used and optionally specify the number of threads to use. If the default thread count of -1 is used, the backend will choose a sensible value based on the available hardware concurrency.

\clearpage
\subsubsection*{Configuration::setCudaBackend}
\label{fn: setCudaBackend}
specify that the CUDA backend should be used


\noindent C++:
\begin{lstlisting}[aboveskip=2pt]
void
Configuration::setCudaBackend(int deviceNumber)
\end{lstlisting}

\noindent C:
\begin{lstlisting}[aboveskip=2pt]
nemo_status_t
nemo_set_cuda_backend(nemo_configuration_t conf, int deviceNumber)
\end{lstlisting}

\noindent Matlab:
\begin{lstlisting}[aboveskip=2pt]
conf.setCudaBackend(deviceNumber)
\end{lstlisting}
\paragraph{Inputs}
\begin{description}
\item[deviceNumber]
\end{description}
Specify that the CUDA backend should be used and optionally specify a desired device. If the (default) device value of -1 is used the backend will choose the best available device.   If the cuda backend (and the chosen device) cannot be used for  whatever reason, an exception is raised.   The device numbering is the numbering used internally by nemo (see  cudaDeviceCount and cudaDeviceDescription). This device  numbering may differ from the one provided by the CUDA driver  directly, since nemo ignores any devices it cannot use. 

\clearpage
\subsubsection*{Configuration::setStdpFunction}
\label{fn: setStdpFunction}
enable STDP and set the global STDP function


\noindent C++:
\begin{lstlisting}[aboveskip=2pt]
void
Configuration::setStdpFunction(const vector<float>& prefire, 
        const vector<float>& postfire, float minWeight, float maxWeight)
\end{lstlisting}

\noindent C:
\begin{lstlisting}[aboveskip=2pt]
nemo_status_t
nemo_set_stdp_function(nemo_configuration_t conf, 
        float prefire[], size_t prefire_len, 
        float postfire[], size_t postfire_len, float minWeight, 
        float maxWeight)
\end{lstlisting}

\noindent Matlab:
\begin{lstlisting}[aboveskip=2pt]
conf.setStdpFunction(prefire, postfire, minWeight, maxWeight)
\end{lstlisting}
\paragraph{Inputs}
\begin{description}
\item[prefire] STDP function values for spikes arrival times before the postsynaptic firing, starting closest to the postsynaptic firing
\item[postfire] STDP function values for spikes arrival times after the postsynaptic firing, starting closest to the postsynaptic firing
\item[minWeight] Lowest (negative) weight beyond which inhibitory synapses are not potentiated
\item[maxWeight] Highest (positive) weight beyond which excitatory synapses are not potentiated
\end{description}
The STDP function is specified by providing the values sampled at integer cycles within the STDP window.

\clearpage
\subsubsection*{Configuration::backendDescription}
\label{fn: backendDescription}
Description of the currently selected simulation backend


\noindent C++:
\begin{lstlisting}[aboveskip=2pt]
std::string
Configuration::backendDescription()
\end{lstlisting}

\noindent C:
\begin{lstlisting}[aboveskip=2pt]
nemo_status_t
nemo_backend_description(nemo_configuration_t conf, const char** description)
\end{lstlisting}

\noindent Matlab:
\begin{lstlisting}[aboveskip=2pt]
description = conf.backendDescription()
\end{lstlisting}
\paragraph{Outputs}
\begin{description}
\item[description] Textual description of the currently selected backend
\end{description}
The backend can be changed using setCudaBackend or setCpuBackend

\clearpage
\subsubsection*{Configuration::resetConfiguration}
\label{fn: resetConfiguration}
Replace configuration with default configuration


\noindent C++:
\begin{lstlisting}[aboveskip=2pt]
void
Configuration::resetConfiguration()
\end{lstlisting}

\noindent C:
\begin{lstlisting}[aboveskip=2pt]
nemo_status_t
nemo_reset_configuration(nemo_configuration_t conf)
\end{lstlisting}

\noindent Matlab:
\begin{lstlisting}[aboveskip=2pt]
conf.resetConfiguration()
\end{lstlisting}

\clearpage
\subsection{Simulation class}
\label{Simulation}
A simulation is created from a network and a configuration object. The simulation is run by stepping through it, providing stimulus as appropriate. It is possible to read back synapse data at run time. The simulation also maintains a timer for both simulated time and wallclock time.

\noindent C++:
\begin{lstlisting}[aboveskip=2pt]
Simulation*
simulation(const Network&, const Configuration&)
\end{lstlisting}

\noindent C:
\begin{lstlisting}[aboveskip=2pt]
nemo_simulation_t
nemo_new_simulation(nemo_network_t net, nemo_configuration_t conf)
 
nemo_delete_simulation(nemo_simulation_t sim)
\end{lstlisting}

\noindent Matlab:
\begin{lstlisting}[aboveskip=2pt]
sim = nemoSimulation(net, conf)
\end{lstlisting}
\paragraph{Functions}
\begin{itemize}
\item \hyperref[fn: step]{step}
\item \hyperref[fn: applyStdp]{applyStdp}
\item \hyperref[fn: getTargets]{getTargets}
\item \hyperref[fn: getDelays]{getDelays}
\item \hyperref[fn: getWeights]{getWeights}
\item \hyperref[fn: getPlastic]{getPlastic}
\item \hyperref[fn: elapsedWallclock]{elapsedWallclock}
\item \hyperref[fn: elapsedSimulation]{elapsedSimulation}
\item \hyperref[fn: resetTimer]{resetTimer}
\item \hyperref[fn: createSimulation]{createSimulation}
\item \hyperref[fn: destroySimulation]{destroySimulation}
\end{itemize}
\clearpage
\subsubsection*{Simulation::step}
\label{fn: step}
run simulation for a single cycle (1ms)


\noindent C++:
\begin{lstlisting}[aboveskip=2pt]
const vector<unsigned>&
Simulation::step(const vector<unsigned>& fstim, 
        const vector<unsigned>& istim_nidx, 
        const vector<float>& istim_current)
\end{lstlisting}

\noindent C:
\begin{lstlisting}[aboveskip=2pt]
nemo_status_t
nemo_step(nemo_simulation_t sim, 
        unsigned fstim[], size_t fstim_len, unsigned istim_nidx[], 
        float istim_current[], size_t istim_current_len, 
        unsigned* fired[], size_t* fired_len)
\end{lstlisting}

\noindent Matlab:
\begin{lstlisting}[aboveskip=2pt]
fired = sim.step(fstim, istim_nidx, istim_current)
\end{lstlisting}
\paragraph{Inputs}
\begin{description}
\item[fstim] An optional list of neurons, which will be forced to fire this cycle
\item[istim_nidx] An optional list of neurons which will be given input current stimulus this cycle
\item[istim_current] The corresponding list of current input
\end{description}
\paragraph{Outputs}
\begin{description}
\item[fired] Neurons which fired this cycle
\end{description}

\clearpage
\subsubsection*{Simulation::applyStdp}
\label{fn: applyStdp}
update synapse weights using the accumulated STDP statistics


\noindent C++:
\begin{lstlisting}[aboveskip=2pt]
void
Simulation::applyStdp(float reward)
\end{lstlisting}

\noindent C:
\begin{lstlisting}[aboveskip=2pt]
nemo_status_t
nemo_apply_stdp(nemo_simulation_t sim, float reward)
\end{lstlisting}

\noindent Matlab:
\begin{lstlisting}[aboveskip=2pt]
sim.applyStdp(reward)
\end{lstlisting}
\paragraph{Inputs}
\begin{description}
\item[reward] Multiplier for the accumulated weight change
\end{description}

\clearpage
\subsubsection*{Simulation::getTargets}
\label{fn: getTargets}
return the targets for the specified synapses


\noindent C++:
\begin{lstlisting}[aboveskip=2pt]
const vector<unsigned>&
Simulation::getTargets(const vector<uint64_t>& synapses)
\end{lstlisting}

\noindent C:
\begin{lstlisting}[aboveskip=2pt]
nemo_status_t
nemo_get_targets(nemo_simulation_t sim, 
        uint64_t synapses[], size_t synapses_len, unsigned* targets[])
\end{lstlisting}

\noindent Matlab:
\begin{lstlisting}[aboveskip=2pt]
targets = sim.getTargets(synapses)
\end{lstlisting}
\paragraph{Inputs}
\begin{description}
\item[synapses] synapse ids (as returned by addSynapse)
\end{description}
\paragraph{Outputs}
\begin{description}
\item[targets] indices of target neurons
\end{description}

\clearpage
\subsubsection*{Simulation::getDelays}
\label{fn: getDelays}
return the conductance delays for the specified synapses


\noindent C++:
\begin{lstlisting}[aboveskip=2pt]
const vector<unsigned>&
Simulation::getDelays(const vector<uint64_t>& synapses)
\end{lstlisting}

\noindent C:
\begin{lstlisting}[aboveskip=2pt]
nemo_status_t
nemo_get_delays(nemo_simulation_t sim, 
        uint64_t synapses[], size_t synapses_len, unsigned* delays[])
\end{lstlisting}

\noindent Matlab:
\begin{lstlisting}[aboveskip=2pt]
delays = sim.getDelays(synapses)
\end{lstlisting}
\paragraph{Inputs}
\begin{description}
\item[synapses] synapse ids (as returned by addSynapse)
\end{description}
\paragraph{Outputs}
\begin{description}
\item[delays] conductance delays of the specified synpases
\end{description}

\clearpage
\subsubsection*{Simulation::getWeights}
\label{fn: getWeights}
return the weights for the specified synapses


\noindent C++:
\begin{lstlisting}[aboveskip=2pt]
const vector<float>&
Simulation::getWeights(const vector<uint64_t>& synapses)
\end{lstlisting}

\noindent C:
\begin{lstlisting}[aboveskip=2pt]
nemo_status_t
nemo_get_weights(nemo_simulation_t sim, 
        uint64_t synapses[], size_t synapses_len, float* weights[])
\end{lstlisting}

\noindent Matlab:
\begin{lstlisting}[aboveskip=2pt]
weights = sim.getWeights(synapses)
\end{lstlisting}
\paragraph{Inputs}
\begin{description}
\item[synapses] synapse ids (as returned by addSynapse)
\end{description}
\paragraph{Outputs}
\begin{description}
\item[weights] weights of the specified synapses
\end{description}

\clearpage
\subsubsection*{Simulation::getPlastic}
\label{fn: getPlastic}
return the boolean plasticity status for the specified synapses


\noindent C++:
\begin{lstlisting}[aboveskip=2pt]
const vector<unsigned char>&
Simulation::getPlastic(const vector<uint64_t>& synapses)
\end{lstlisting}

\noindent C:
\begin{lstlisting}[aboveskip=2pt]
nemo_status_t
nemo_get_plastic(nemo_simulation_t sim, 
        uint64_t synapses[], size_t synapses_len, unsigned char* plastic[])
\end{lstlisting}

\noindent Matlab:
\begin{lstlisting}[aboveskip=2pt]
plastic = sim.getPlastic(synapses)
\end{lstlisting}
\paragraph{Inputs}
\begin{description}
\item[synapses] synapse ids (as returned by addSynapse)
\end{description}
\paragraph{Outputs}
\begin{description}
\item[plastic] plasticity status of the specified synpases
\end{description}

\clearpage
\subsubsection*{Simulation::elapsedWallclock}
\label{fn: elapsedWallclock}



\noindent C++:
\begin{lstlisting}[aboveskip=2pt]
unsigned long
Simulation::elapsedWallclock()
\end{lstlisting}

\noindent C:
\begin{lstlisting}[aboveskip=2pt]
nemo_status_t
nemo_elapsed_wallclock(nemo_simulation_t sim, unsigned long* elapsed)
\end{lstlisting}

\noindent Matlab:
\begin{lstlisting}[aboveskip=2pt]
elapsed = sim.elapsedWallclock()
\end{lstlisting}
\paragraph{Outputs}
\begin{description}
\item[elapsed] number of milliseconds of wall-clock time elapsed since first simulation step (or last timer reset)
\end{description}

\clearpage
\subsubsection*{Simulation::elapsedSimulation}
\label{fn: elapsedSimulation}



\noindent C++:
\begin{lstlisting}[aboveskip=2pt]
unsigned long
Simulation::elapsedSimulation()
\end{lstlisting}

\noindent C:
\begin{lstlisting}[aboveskip=2pt]
nemo_status_t
nemo_elapsed_simulation(nemo_simulation_t sim, unsigned long* elapsed)
\end{lstlisting}

\noindent Matlab:
\begin{lstlisting}[aboveskip=2pt]
elapsed = sim.elapsedSimulation()
\end{lstlisting}
\paragraph{Outputs}
\begin{description}
\item[elapsed] number of milliseconds of simulation time elapsed since first simulation step (or last timer reset)
\end{description}

\clearpage
\subsubsection*{Simulation::resetTimer}
\label{fn: resetTimer}
reset both wall-clock and simulation timer


\noindent C++:
\begin{lstlisting}[aboveskip=2pt]
void
Simulation::resetTimer()
\end{lstlisting}

\noindent C:
\begin{lstlisting}[aboveskip=2pt]
nemo_status_t
nemo_reset_timer(nemo_simulation_t sim)
\end{lstlisting}

\noindent Matlab:
\begin{lstlisting}[aboveskip=2pt]
sim.resetTimer()
\end{lstlisting}

\clearpage
\subsubsection*{Simulation::createSimulation}
\label{fn: createSimulation}
Initialise simulation data


\noindent C++:
\begin{lstlisting}[aboveskip=2pt]
void
Simulation::createSimulation()
\end{lstlisting}

\noindent C:
\begin{lstlisting}[aboveskip=2pt]
nemo_status_t
nemo_create_simulation(nemo_simulation_t sim)
\end{lstlisting}

\noindent Matlab:
\begin{lstlisting}[aboveskip=2pt]
sim.createSimulation()
\end{lstlisting}
Initialise simulation data, but do not start running. Call step to run simulation. The initialisation step can be time-consuming.

\clearpage
\subsubsection*{Simulation::destroySimulation}
\label{fn: destroySimulation}
Stop simulation and free associated data


\noindent C++:
\begin{lstlisting}[aboveskip=2pt]
void
Simulation::destroySimulation()
\end{lstlisting}

\noindent C:
\begin{lstlisting}[aboveskip=2pt]
nemo_status_t
nemo_destroy_simulation(nemo_simulation_t sim)
\end{lstlisting}

\noindent Matlab:
\begin{lstlisting}[aboveskip=2pt]
sim.destroySimulation()
\end{lstlisting}
The simulation can have a significant amount of memory associated with it. Calling destroySimulation frees up this memory.

\clearpage