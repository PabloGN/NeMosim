\documentclass[a4paper]{article}

\usepackage{graphicx}
\usepackage{listings}
\usepackage{hyperref}

\setlength{\parindent}{0pt} 
\setlength{\parskip}{2ex}

\lstset{
	tabsize=4,
	%basicstyle=\small \ttfamily,
	%basicstyle={\rmfamily},
	%basicstyle={\ttfamily},
	%numbers=left,
	numberstyle=\tiny,
	stepnumber=1,
	numbersep=5pt,
	frame=none,
}

\lstnewenvironment
	{matlab}
	{ 	\lstset{
			language=Matlab,
			%basicstyle=\small \ttfamily,
			frame=single,
			numbers=none
	}	}
	{ }

\lstnewenvironment
	{shell}
	{ 	\lstset{
			%language=Matlab,
			%basicstyle=\small \ttfamily,
			basicstyle={\ttfamily},
			frame=off,
			numbers=none
	}	}
	{ }

\newcommand{\nemo}{\texttt{NeMo}}
\newcommand{\code}[1]{\texttt{#1}}
\newcommand{\command}[1]{\texttt{#1}}
\newcommand{\file}[1]{\texttt{#1}}
\newcommand{\directory}[1]{\texttt{#1}}
\newcommand{\library}[1]{\texttt{#1}}

% rst2latex leaves in title references
\newcommand{\titlereference}[1]{\textsl{#1}}

% rst2latex also sets this on inside the document
\newlength{\locallinewidth}

\input{title}

\begin{document}

\maketitle

\section{Overview}

nemo is a library for discrete-time simulation of spiking neural networks.
It runs simulations on CUDA-enabled GPUs (device generation $\geq$ 1.2).
The library provides a C++ and a C interface. 

\section{Installation}

\subsection{General dependencies}

\begin{itemize}
	\item CUDA toolkit $\geq$ 2.1. The SDK is not required.
	\item Boost
	\item CMake $\geq$ 2.8
\end{itemize}

Note that the CUDA compiler driver \command{nvcc} is not compatible with all versions
of \command{gcc}, especially newer versions of \command{gcc}.

For Boost, mostly header-only libraries are required (\library{shared\_ptr}, \library{tuples},
\library{random}, \library{date\_time}), so a full install/build is generally not required. 
Note, however, that some versions of Visual C++ requires \library{date\_time} to be compiled as
a proper library. 
The \library{date\_time} library is only required for simulation timing,
	and can be disabled in the cmake configuration stage via the \code{INCLUDE\_TIMING\_API} variable.

\subsection{Unix}

The source package can be built using cmake. The basic procedure is 

% TODO: use code environment
\begin{shell}
cd <nemo-directory>
mkdir build
cd build
cmake ..
make
sudo make install
\end{shell}

By default this installs headers to \directory{/usr/include/nemo} and the
library to \directory{/usr/lib}. 

\subsection{Windows}

A precompiled library is available (NSIS installer).
Both header and library files are stored in \directory{c:\textbackslash Program Files\textbackslash nemo-<version>}.
To use, make sure this directory is on the include path and on the library path
and then include either \file{nemo.hpp} or \file{nemo.h}.

The source package can be built by using \command{cmake-gui} to generate an MSVC project file,
	and then building from within MSVC:

\begin{enumerate}
	\item start \command{cmake-gui}
	\item enter the source and target directories (these should be different)
	\item configure
	\item generate
	\item open the resulting project file and build from within visual studio
\end{enumerate}

Builds via cygwin or msys/mingw have not been tested.

\section{Usage}

\subsection{General}

Both the C++ and C interfaces follow the same general usage pattern:

\begin{enumerate}
	\item Create a network object (\code{nemo::Network}) and add neurons and synapses
	\item Create a configuration object (\code{nemo::Configuration}) and configure as appropriate
	\item Create a simulation object (\code{nemo::Simulation}) from a network and a configuration and run the simulation
\end{enumerate}

For the C++ API include the header file \file{nemo.hpp}.
Errors are reported via exceptions, all of which are subclasses of \code{std::exception}.

For the C API, include the header file \file{nemo.h}. 
In the C API the network, configuration, and simulation objects are controlled
via opaque pointers.
The objects are generated with methods 
\code{nemo\_new\_x} (\code{x} $=$ \code{network}, \code{configuration}, or \code{simulation}),
and should be explicitly destroyed with the corresponding methods 
\code{nemo\_delete\_x}.
The class methods are accessed with functions with names prefixed with \code{nemo\_},
	all of which take the relevant opaque pointer as the first parameter.
The C API communicates errors via error codes.
Most functions return a value of type \code{nemo\_status\_t},
	which will be \code{NEMO\_OK} if everything went fine and some other value otherwise.

The header files \file{nemo.hpp} and \file{nemo.h} document the API functions in more detail.
The files in the example directory in the distribution shows how to use the library.

\subsection{Matlab}

The Matlab interface closely mirrors the underlying C++ interface.
The three objects nemoNetwork, nemoConfiguration and nemoSimulation corresponds
to the relevant C++ objects,
	and provides the same methods,
	with only small variations in usage.
The class methods are documented using Matlab's help system,
	i.e. via calls such as \code{help nemoNetwork} and \code{help addNeuron}.

Errors in the simulation backend result in Matlab errors,
	(i.e. as when \code{error} is called in a script).

The Matlab path must contain the directory with the m-files defining the three classes and the MEX
library that interfaces with libnemo.
On Windows this directory defaults to \file{C:\textbackslash Program Files\textbackslash nemo-<version>\textbackslash Matlab},
and on Linux to \file{/usr/share/nemo/matlab}.
Use \code{addpath} from within Matlab to set the path.

Additionally, the nemo library and the CUDA runtime library needs to be on the
system path (which is different from the Matlab path).
If this is not the case Matlab will issue a rather unhelpful message about the MEX-file being invalid.

\subsection{Simulation model}

The simulation is discrete-time with a fixed one millisecond step size.
Neurons are based on the Izhikevich neuron model \cite{izhikevich2003simple_model}.
Within each step the following actions take place in a fixed order:

\begin{enumerate}
	\item Compute accumulated current for incoming spikes;
	\item Update the neuron state;
	\item Determine if any neurons fired. The user can specify neurons which should be forced to fire at this point;
	\item Update the state of the fired neurons
	\item Accumulate STDP statistics, if STDP is enabled
\end{enumerate}

The state update step uses the Euler method with a step size of 0.25ms. 

The weights are stored in a fixed-point format internally for performance reasons.
The specific fixed-point format in use depends on the range of weights in the input network.
The current accumulation uses saturating arithmetic to avoid overflow.
Neuron parameters are stored in single-precision floating point.

The maximum delay is currently limited to 64ms.

\subsection{STDP model}
\label{stdp}

\nemo\ supports spike-timing dependant plasticity,
	i.e. synapses can change during simulation depending on the temporal relationship 
between the firing of the pre- and post-synaptic neurons. 
To make use of STDP the user must first enable STDP globally by specifying an STDP function,
and then enable plasticity for each synapse as appropriate.
A single STDP function is applied to the whole network.

Synapses can be either potentiated or depressed.
With STDP enabled,
	the simulation accumulates a weight change
	which is the sum of potentiation and depression for each synapse. 
Potentiation always moves the synaptic weight away from zero,
	which for excitatory synapses is more positive,
	and for inhibitory synapses is more negative.
Depression always moves the synapses weight towards zero.
The accumulation of potentiation and depression statistics takes place every cycle,
	but the modification of the weight only takes place when explicitly requested.

Generally a synapse is potentiated if a spike arrives shortly before the postsynaptic neuron fires. 
Conversely, if a spike arrives shortly after the postsynaptic firing the synapse is depressed. 
Also, the effect of either potentiation or depression generally weakens as the time difference, $dt$, 
	between spike arrival and firing increases. 
Beyond certain values of $dt$ before or after the firing, STDP has no effect. 
These limits for $dt$ specify the size of the STDP window.

The user can specify the following aspects of the STDP function:

\begin{itemize}
	\item the size of the STDP window;
	\item what values of $dt$ cause potentiation and which cause depression;
	\item the strength of either potentiation or depression for each value of $dt$, 
		i.e. the shape of the STDP function;
	\item maximum weight of plastic excitatory synapses; and
	\item minimum weight of plastic inhibitory synapses.
\end{itemize}

Since the simulation is discrete-time,
	the STDP function can be specified by providing values of the underlying function
	sampled at integer values of $dt$.
For any value of $dt$ a positive value of the function denotes potentiation, 
	while a negative value denotes depression.
The STDP function is described using two vectors: 
	one for spike arrivals \emph{before} the postsynaptic firing (pre-post pair),
	and one for spike arrivals \emph{after} the postsynaptic firing (post-pre pair).
The total length of these two vectors is the size of the STDP window.
The typical scheme is to have positive values for pre-post pairs and negative values for post-pre pairs,
but other schemes can be used.

When accumulating statistics a pairwise nearest-neighbour protocol is used.
For each postsynaptic firing potentiation and depression statistics are updated based
on the nearest pre-post spike pair (if any inside STDP window) 
and the nearest post-pre spike pair (if any inside the STDP window).

Excitatory synapses are never potentiated beyond the user-specified maximum weight,
and are never depressed below zero.
Likewise, inhibitory synapses are never potentiated beyond the  user-specified minimum weight,
and are never depressed above zero.
Synapses can thus be deactivated, but never change from excitatory to inhibitory or vice versa.

% TODO: max size of window


%As a practical example, consider an asymmetric STDP scheme specified using the following functions for LTP and LTD:
%\[ stdp(dt) = \alpha e^{-dt / \tau} \]
%where $\alpha$ specifies a scaling and $\tau$ the size of either the potentiation or depression window. For LTP, let $\alpha = 1.0, \tau=20$, and for LTD let $alpha=0.8, \tau=20$. This STDP function is specified using two vectors like this:


\bibliographystyle{abbrv}
\bibliography{manual}

\end{document}
